\documentclass{article}
\usepackage[pdftex,a4paper]{geometry}
\usepackage{graphicx}
\usepackage{times}
\usepackage[francais]{babel}
\geometry{top=0.8cm,bottom=0.8cm,left=1cm,right=1cm}
\pagestyle{empty}
\begin{document}

\begin{center}
\Large\textbf{
Projet de Syst\`emes d'Exploitation 2012}
\end{center}

\section{Pr\'esentation}

Le but de ce projet est la mise en \oe uvre d'un simulateur (l\'eger) des
composantes mat\'erielles principales d'un ordinateur usuel fonctionnant sous
un syst\`eme d'exploitation multit\^ache, en mettant l'accent sur l'unit\'e
centrale et sur une impl\'ementation dans le noyau de la communication et
synchronisation entre processus.
Ce simulateur offre des \'el\'ements de programmation
machine (assembleur fictif simplifi\'e, gestion d'interruptions) et des 
\'el\'ements de programmation syst\`eme (gestion de plusieurs processus,
m\'emoire partag\'ee et s\'emaphores).

Le noyau est le processus (sim) de PID 0, et offre aux processus des services
sous la forme de plusieurs interruptions~: \'elire le processus suivant qui est
pr\^et (\texttt{int1}, tr\`es simple round robin), terminer le processus en cours
(\texttt{int2} -- un peu comme \texttt{exit()}), op\'eration sur les
s\'emaphores (\texttt{int4}), entr\'ees-sorties "console" (\texttt{int5} et
\texttt{int6}) ; l'interruption \texttt{int3} est r\'eserv\'ee pour
l'ordonnanceur.

Tous les processus sont charg\'es au d\'ebut (il n'y a pas de shell) et
ex\'ecut\'es un par un pas \`a pas jusqu'\`a leur fin our jusqu'\`a des
requ\^etes d'op\'eration de s\'emaphores bloquantes. Chaque
processus dispose de sa propre tranche de m\'emoire vive (sim) "virtuelle"
de longueur fixe, y compris le noyau (qui peut \'egalement acc\'eder aux
tranches des processus.

De surcro\^\i t, il y a une autre tranche de m\'emoire -- partag\'ee entre
tous les processus.

Pour chaque type de m\'emoire il y a des instructions assembleur
sp\'ecifiques, pour simplifier leur manipulation (e.g. \texttt{LDMEM},
\texttt{LDSHM}, etc.) 

Le cadre de programmation, d\'ecrit plus bas, vous est fourni en \'etat de marche
et vous devez l'enrichir. Des classes repr\'esentent chaque \'el\'ement
d\'ecrit plus haut~; la m\'emoire vive est log\'ee en entier dans un seul
fichier, avec des offset pour chaque tranche~; les textes des programmes (sim)
(des processus) sont des simple strings, g\'er\'es par des classes
d\'edi\'ees, la m\'emoire \'etant uniquement pour les donn\'ees, pour
simplifier le tout.  

Ce cadre permet donc d\'ej\`a tel qu'il est la compilation et son ex\'ecution,
pour comprendre son fonctionnement et tester les fonctionnalit\'es d\'ej\`a
fournies, \`a l'aide d'exemples inclus. 


\`A la fin de ce document une section d\'etaille votre travail \`a
faire.

\section{Tester un peu}

\begin{itemize}
\item d\'esarchivez et compilez (\texttt{make nom=System}, qui contient le \texttt{main()})
\item lancez le tout (\texttt{System.run}) et regardez l'\'ecran, et/ou
  gardez-en une copie dans un fichier \'egalement (par exemple avec
  \texttt{tee}, ou simplement en dirigeant la sortie dans le fichier --
  n'oubliez pas de r\'ecup\'erer la sortie erreur standard \'egalement).
\item le tout doit finir, avec un dump de la m\'emoire 
\end{itemize}

\section{D\'eroulement d\'etaill\'e pas-\`a-pas}

Au d\'emarrage du simulateur, apr\`es construction des objets et chargement du
code de tous les programmes (noyau et processus normaux), c'est le noyau qui est
d\'emarr\'e (\texttt{Proc\_0\_Program.asm}). Celui-ci
\begin{itemize}
\item arrange sa m\'emoire, en y mettant 
\begin{itemize}
 \item vecteur d'interruptions (cod\'e en dur dans le texte du programme, \$prep), 
 \item tables \'etats processus (remplies avec \texttt{readytorun}),
 \item vecteur avec adresses de d\'ebut des vecteurs de requ\^etes de semop
   par processus
\end{itemize}
\item se branche sur \texttt{int1}
\begin{itemize}
 \item cherche le premier processus P en \'etat \texttt{readytorun}
\item change son \'etat en \texttt{running} dans la table
\item se branche sur le code de P (passant en mode utilisateur) et commence
  ainsi son ex\'ecution (instruction \texttt{LDPSW} qui charge le PSW --
  Program Status Word -- l'\'etat de l'unit\'e centrale (les registres, etc.)
  correspondant au processus P)
\end{itemize}
\end{itemize}

Lorsque le processus P en cours fait une requ\^ete noyau (donc appelle une
interruption, avec l'instruction \texttt{CLINT})~:

\begin{itemize}
\item l'\'etat de l'unit\'e centrale est sauvegard\'e automatiquement
  (i.e. le PSW)
\item le pid de P est not\'e dans la m\'emoire vive du noyau
\item on se branche \`a nouveau sur le code du noyau, dans l'interruption en
  question (et en mode
  ma\^\i tre), qui, dans la plupart des cas
\begin{itemize}
\item pr\'epare les param\'etres de la requ\^ete 
\item l'effectue
\item se branche \'eventuellement sur \texttt{int1} pour essayer de trouver un
  autre   processus \`a ex\'ecuter 
\end{itemize}
\end{itemize}

\section{Structure}

La partie programmation s'appuie sur plusieurs classes qui vous sont fournies.

Les fichiers \texttt{Board.h} et \texttt{Board.cxx} contiennent les classes
qui repr\'esentent la m\'emoire vive (Memory) et un prototype d'unit\'e
centrale (BaseCPU) avec ses registres (g\'en\'eraux, sp\'eciali\'es,
etc.). Ces fichiers sont en \'etat d'utilisation tels quels et n'ont pas
n\'ecessairement besoin d'apport pour ce projet. 

Les fichiers \texttt{CPU.h} et \texttt{CPU.cxx} repr\'esentent l'unit\'e
centrale de notre syst\`eme. Nous ne simulons pas la notion de bus avec une
classe s\'epar\'ee~: la classe BaseCPU "poss\`ede" \'egalement l'objet
repr\'esentant la m\'emoire vive. 

\section{M\'emoire}

La m\'emoire vive du simulateur est impl\'ement\'ee \`a l'aide d'un seul
fichier Unix normal, et donc permet \'egalement un acc\`es concurrent si
besoin est, \`a l'aide m\^eme de plusieurs processus (vrais -- Unix). Le code
qui vous est fourni n'en a toutefois qu'un seul. 

Chaque processus du simulateur dispose d'une tranche de $1000$ entiers (donc
$1000 *$\texttt{sizeof(int)} octets), qu'il voit de $0$ a $999$ (m\'emoire
"virtuelle"). Ces tranches sont en fait simplement log\'ees une apr\`es
l'autre dans le grand fichier unique (adresses en mots -- taille d'un \texttt{int}):

{\small
\begin{verbatim}
AddrFic  AddrProc   MemProc      Utilisation
-----------------Debut du fichier-------------------

   0       0       0 (noyau)     <speciale>
   10     10       0                ...
   ....   ....     0
   990   990       0 

- - - - - - - - - - - - fin tranche - - - - - - - - -

  1000     0       1         <mini-zone-u-dot>      
  1010    10       1         
   ....   ....     1                ...
  1200   200       1    <debut zone disponible>
   ....   ....     1                ...
  1990   990       1 <start pile (croit vers debut)>

- - - - - - - - - - - - fin tranche - - - - - - - - -

  2000     0       2         <mini-zone-u-dot>      
  1010    10       2          <detail +bas>
  1020    20       2                ...
   ....   ....     2
  1200   200       2    <zone disponible>
   ....   ....     2                ...
  1990   990       2 <start pile (croit vers debut)>

- - - - - - - - - - - - fin tranche - - - - - - - - -
       ...         ...       ...
- - - - - - - - - - - - fin tranche - - - - - - - - -
   Memoire Partagee (visible par tous les processus)
-----------------Fin du fichier-------------------
\end{verbatim}}

Donc la tranche du processus simulateur $p$ commence dans le fichier m\'emoire
\`a l'offset r\'eel en octets $p*1000*$\texttt{sizeof(int)}.

Ces constantes (parmi d'autres similaries) sont d\'efinies au d\'ebut du fichier \texttt{CPU.h}: 

\begin{verbatim}
#define MEMORYSIZEPERPROCESS 1000 
#define KERNELSAVECURRENTPROCID 0
#define KERNELPID               0 
#define PROCPSWFRAMESTART       0
\end{verbatim}

\section{L'unit\'e centrale}

\begin{itemize}
\item a un jeu de registres \`a usage g\'en\'eral (les \texttt{Rn})
\item a des registres sp\'ecialis\'es~: le mode (user ou master), le pointeur
  de pile, le compteur ordinal (program counter), le compteur de processus
  (process counter), le drapeau d'interruptions, et les registres d'entr\'ee
  pour les communications.
\begin{verbatim}
class BaseCPU {
protected:
    InstructionParser iPrs;
    StackPointer     spReg; 
    ProgramCounter   pcReg;
    ProcessCounter   prReg; 
    RunMode          mdReg;  // user or master
    RegisterSet      genReg;
    Register         inReg0,inReg1; // for interCPU communication (e.g. CPU<->DMA) 
    MemoryTable      mem;
    Memory           sharedMem;
    ProcessTable     proc;
    string           cpuId;
    ostream         &logStream;
    ostream         &consoleOutputStream;
    istream         &consoleInputStream;
    ConsoleInOut     consoleInOut;
    bool             qRun;
    bool             qInterruptible;
public:
    BaseCPU(const string &id, ostream  &log, const int nGenReg, const int nProc);
    virtual void dumpReg(ostream &os, bool qRegAsWell = false);
    virtual void execute(const Instruction &) throw(nsSysteme::CExc);
    virtual void run() throw(nsSysteme::CExc);
    virtual void interrupt(const int, bool) throw(nsSysteme::CExc);
    virtual void pendingIntIfAny() throw(nsSysteme::CExc);
};
\end{verbatim}
\item la boucle principale d'ex\'ecution se trouve dans la m\'ethode
  \texttt{run()}
\begin{verbatim}
void BaseCPU::run() throw(CExc) {
    for(iTick = uTick = 0; qRun && iTick < 10000; uTick += (mdReg . getVal() != 1),iTick++) {
      execute(pcReg . fetch(proc[prReg . getVal()] . programText));
      if(qInterruptible) { pendingIntIfAny(); }
    }
    logStream << "CPU " << cpuId << " is now fully stopped, total " << iTick << " instructions overall.\n";
}
\end{verbatim}
\item sait ex\'ecuter les instructions qui sont d\'ecrites en d\'etail dans \texttt{Board.h} 

\begin{verbatim}
enum InstructionType {

NOPER=0,
DMPRG, // ex: DMPRG              ; dump content of all registers to logstream
MSTRM, // ex: MSTRM              ; MD <- MasterMode
USERM, // ex: USERM              ; MD <- UserMode
NINTR, // ex: NINTR              ; disable interrupts
INTRP, // ex: INTRP              ; enable interrupts
SETRI, // ex: SETRI R2 254       ; R2 <- 254
SETRG, // ex: SETRG R2 R3        ; R2 <- R3
SETPR, // ex: SETPR R10          ; PR <- R10 (only in mastermode)
CPPRG, // ex: CPPRG R2           ; R2 <- PR  
ADDRG, // ex: ADDRG R2 R3 R4     ; R2 <- R3 + R4  
SUBRG, // ex: SUBRG R2 R3 R4     ; R2 <- R3 - R4  
PSHRG, // ex: PSHRG R5 R4        ; mem[(SP - R5) * sizeof(int)] <- R4
POPRG, // ex: POPRG R5 R7        ; R7 <- mem[(SP + R5) * sizeof(int)];
JMABS, // ex: JMABS R1           ; PC <- R1 
JMPTO, // ex: JMPTO R1           ; PC <- PC + R1 
JZERO, // ex: JZERO R12 R5       ; if(R12 == 0) { PC <- PC + R5 }
JNZRO, // ex: JNZRO R12 R5       ; if(R12 != 0) { PC <- PC + R5 }
JMBSI, // ex: JMBSI 23           ; PC <- 23 
JMTOI, // ex: JMTOI 24           ; PC <- PC + 24 
JZROI, // ex: JZROI R12 25       ; if(R12 == 0) { PC <- PC + 25 }
JNZRI, // ex: JNZRI R12 26       ; if(R12 != 0) { PC <- PC + 26 }
LDMEM, // ex: LDMEM R8 R21       ; R21     <- mem[R8] (for current proc memory space)
STMEM, // ex: STMEM R8 R21       ; mem[R8] <- R21  (for current proc memory space)
LDSHM, // ex: LDSHM R8 R21       ; R21     <- sharedMem[R8] (for all procs shared memory space)
STSHM, // ex: STSHM R8 R21       ; sharedMem[R8] <- R21  (for all procs shared memory space)
LDPSW, // ex: LDPSW R3           ; <SP,PR,PC,MD,R0,R1,...> <- memOfProc[R3][PROCPSWFRAMESTART]
SPSWR, // ex: SPSWR R3 R2 R4     ; memOfProc[R3][PROCPSWFRAMESTART+offset4.R2] <- R4 (master only)
LDPRM, // ex: LDPRM R3 R8 R21    ; R21     <- memOfProc[R3][R8] (master only)
STPRM, // ex: STPRM R3 R8 R21    ; memOfProc[R3][R8] <- R21 (master only)
WKCPU, // ex: WKCPU R3 R5 R6     ; CPU[R3].I0 <- R5 ; CPU[R3].I1 <- R6 ;  signal CPU[R3]
GETI0, // ex: GETI0 R5           ; R5 <- I0 (the only way to read I0 (after a WKCPU))
GETI1, // ex: GETI1 R5           ; R5 <- I1 (the only way to read I1 (after a WKCPU))
CLLSB, // ex: CLLSB R7 R5        ; SP <- SP - R7; mem[SP * sizeof(int)] <- PC; PC <- R5
RETSB, // ex: RETSB R6           ; PC <- mem[SP * sizeof(int)] ; SP <- SP + R6
CLINT, // ex: CLINT R7           ; PC <- IntVec[R7] (saves the PSW except for the kernel)
SPECP, // ex: SPECP R2 R3 R10 R4 ; special instruction given by R2 with (at most) three args(R3,..)
SCRASH,// ex: SCRASH             ; crash down the system (master only) -- major inconsistency 
SDOWN};// ex: SDOWN              ; shut down system (master only)
\end{verbatim}
\item g\`ere des interruptions avec la m\'ethode \texttt{CPU$::$interrupt()}~:
  une interruption peut donc soit "survenir" (cause externe -- ici ce sera
  principalement l'"alarme" du scheduler), soit
  \^etre appel\'ee par un processus utilisateur (instruction \texttt{CLINT}),
  pour demander un service (sortie console (interruption 5), terminaison de
  programme (interruption 2), etc.).
\item sauvegarde son mot d'\'etat du programme (PSW -- Program Status Word) --
  voir m\'ethode \texttt{CPU$::$interrupt()} dans \texttt{CPU.cxx}.
\end{itemize}

\section{Remarques}

\begin{itemize}
\item les instructions ne manipulent pas des registres
sp\'eciaux par leur nom (i.e. en tant qu'arguments), mais certaines instructions
modifient tout de m\^eme directement les valeurs de ces registres sp\'eciaux,
comme \texttt{SETPR} (en mode ma\^\i tre). 
\item les instructions machine reconnues par la CPU travaillent beaucoup
uniquement  avec des registres et peu avec des
valeur num\'erique (on les appelle des instructions "imm\'ediates": 
\texttt{SETRI}, \texttt{JNZRI}, etc.). 
\item puisqu'on ne
se concentre pas sur des calculs approfondis, les instructions
arithm\'etiques sont tr\`es peu nombreuses (additions, soustractions).
\item toutes les valeurs d'addressage dans la m\'emoire vive (au niveau langage
machine de ce projet) repr\'esentent des entiers (des "word"). Par contre,
au niveau impl\'ementation (donc par exemple dans la classe \texttt{Memory}), les
adresses d\'esignent des octets. Dans la m\'ethode \texttt{CPU$::$execute()}
on prend soin de multiplier par \texttt{wordSize} -- la taille d'un entier
-- les valeurs qu'on r\'ecup\`ere des registres avant d'appeler les
m\'ethodes de la classe  \texttt{Memory} pour l'acc\`es effectif. 
\item pour les sauts
(\texttt{JNZRO}, etc.), le \texttt{PC} pointe d\'ej\`a \`a l'instruction qui
suit le saut, donc ceci est \`a prendre en compte lors du calcul du nombre
d'instructions \`a sauter lors de sauts relatifs. 
\item pour les appels de
proc\'edure on dispose de \texttt{CLLSB} et \texttt{RETSB} dont le premier
argument donne la taille de la zone de param\`etres (avec l'adresse-programme
de retour) sur la pile. Pour empiler/r\'ecup\'erer des arguments, on a
\texttt{PSHRG} et \texttt{POPRG}.
\item le code des programmes qui sont en cours d'ex\'ecution (i.e. les
processus) est stock\'e ailleurs que les donn\'ees -- la classe
\texttt{Memory} sert uniquement pour les donn\'ees.
\item le code ("machine") des programmes est en fait stock\'e sous forme de
cha\^\i nes de caract\`eres (ceci aide \'egalement la lisibilit\'e)
\item le registre \texttt{PC} (program counter) est un indice dans ce tableau
de cha\^\i nes
\item Un Proc\_0 plus lisible se trouve dans le sous-r\'epertoire \texttt{ini}
  -- mais ce programme ne repr\'esente pas ce que la machine virtuelle
  voit. Pour des programmes en assembleur/code machine un peu plus larges il
  est plus facile de travailler de mani\`ere symbolique, au moins pour les
  branchements, et donc le programme du  sous-r\'epertoire \texttt{ini} est
  celui-ci, d'un peu plus "haut niveau". Pour le transformer en "vrai code
  machine" pour notre machine virtuelle il faut utiliser le script Perl
  \texttt{preProcAsm.pl} (en tant que filtre, donc par exemple ainsi
\begin{verbatim}
    cat ini/Proc_0_Program.asm | ./preProcAsm.pl > Proc_0_Program.asm
\end{verbatim}
  
\end{itemize}

\section{Travail \`a faire}

\begin{enumerate}
\item \'Etude g\'en\'erale -- cherchez sur le web et dans des livres des
  renseignements sur les points suivants et r\'edigez
  \textbf{individuellement} ensuite un rapport d'environ
  quatre pages, \textit{\`a rendre pour le 18 novembre 2012, 23h59 heure de Paris}~:
\begin{enumerate}
 \item sch\'ema de base de gestion d'un seul processus dans le noyau
\item sch\'ema de base du noyau (ordonnanceur, gestion m\'emoire
  (virtuelle), communication entre processus, pilotes de mat\'eriel (device
  drivers), etc.)
\item concr\`etement -- d\'efinissez les notions suivantes, et expliquez-en le
  but, les co\^uts, les particularit\'es, leur place dans le cadre du
  syst\`eme d'exploitation, etc.~:
\begin{enumerate}
\item registres de l'unit\'e centrale (CPU) et m\'emoire RAM
\item changement de contexte
\item interruption, appel syst\`eme
\item mode utilisateur et mode ma\^\i tre
\item  espace m\'emoire d'un processus
\item m\'emoire partag\'ee et exclusion mutuelle
\end{enumerate}
\item si/comment les concepts ci-avant sont-ils disponibles en C (ou C++) dans
  les syst\`emes de type Unix, comme Linux, etc. ?
\end{enumerate}
\item Compr\'ehension du code fourni (l'impl\'ementation des instructions
  machine est dans \texttt{CPU$::$execute()}) -- travail en \'equipe, \`a
  rendre ensemble avec le code (voir point suivant), pour le 23 d\'ecembre
  2012, 23h59 heure de Paris~:
\begin{enumerate}
\item Quelle est la diff\'erence entre \texttt{STMEM} et \texttt{STPRM}~?
  Pourquoi vaut-il mieux permettre la seconde instruction uniquement en mode
  ma\^\i tre~? 
  Pourquoi suffit-il pour le noyau de faire \texttt{LDPSW} d'un processus P
  pour "se brancher sur le code de P"~?
\item D\'etaillez pas \`a pas ce que fait \texttt{int2} (comme est d\'ecrit le
  d\'eroulement de chacune des trois autres dans ce document). 
\item D\'etaillez la structure de la tranche de m\'emoire de chaque processus
  (regardez la m\'ethode \texttt{CPU$::$interrupt()} et l'instruction
  \texttt{LDPSW}, ainsi que le code noyau de \texttt{int3},
  et enfin les instructions manipulant \texttt{spReg} ainsi que
  \texttt{Memory$::$setUp()}) 
\item \`A quoi sert la ligne 168 de \texttt{Proc\_0\_Program.asm}~?
\item D\'etaillez la structure de la tranche de m\'emoire du noyau (regardez
  son code dans \texttt{Proc\_0\_Program.asm} et la ligne 
\begin{verbatim}
mem[KERNELPID] . storeAt(KERNELSAVECURRENTPROCID, prReg . getVal());
\end{verbatim}
de \texttt{CPU$::$interrupt()}).
\item Quelles sont les limitations (nombre de processus, nombre de
  s\'emaphores, de semop()s, etc.) pr\'esentes dans ce projet, telles qu'elles
  sont impos\'ees par la structure de la m\'emoire et par les valeurs de ces
  constantes d\'efinies dans le code.
\end{enumerate}
\item Construction code  -- travail en \'equipe, \`a
  rendre ensemble avec le code (voir point pr\'ec\'edent), pour le 23 d\'ecembre
  2012, 23h59 heure de Paris~:
\begin{enumerate}
\item Observez que l'interruption 5 en fait ne peut pas afficher plus d'un
  nombre ou un caract\`ere -- elle n'est pas "vectoris\'ee" -- elle ignore la
  valeur du registre \texttt{R2}, lequel est pourtant cens\'e \^etre un de ses
  param\`etre. La premi\`ere
  t\^ache consiste donc \`a r\'e\'ecrire cette interruption \texttt{int5} pour
  qu'elle soit vectoris\'ee, donc qu'elle prenne bien en compte \texttt{R2}
  \'egalement -- vous devez donc rajouter une boucle, etc.
\item \'Ecrire l'interruption \texttt{int6} qui permet de lire depuis le
  clavier des nombres ou des caract\`eres, donc de mani\`ere compl\'ementaire
  \`a ce que fait l'interruption \texttt{int5}.
\item \'Elargir les programmes \texttt{Proc\_1\_Program.asm} et
  \texttt{Proc\_2\_Program.asm} (qui forment un couple
  producteur-consommateur) pour faire \'ecrire les nombres
  de 1 a 10 en m\'emoire partag\'ee depuis l'addresse 11 \'a l'adresse 20, et
  les faire lire, mais en \'ecrivant un nombre \`a la fois dans la zone critique et en rel\^achant
  donc les s\'emaphores. (n'oubliez pas que vous avez plusieurs s\'emaphores
  \`a votre disposition)
\item \'Etendre l'infrastructure de ce syst\`eme (en \'ecrivant du
  C++ et aussi en am\'eliorant ensuite \texttt{Proc\_0\_Program.asm}) 
\begin{itemize}
\item  rajouter la possibilit\'e d'avoir des nombres
  al\'eatoires, avec une lecture memory-mapped dans le noyau, similaire aux
  entr\'ees-sorties console (bien entendu, utilisnt une autre adresse), 
\item augmenter le
  scheduleur pour le randomiser ensuite, 
\item rajouter une gestion CPU multi-core
\item \dots
\end{itemize}
\end{enumerate}
\end{enumerate}
\end{document}
\textbf{\'Ech\'eances~:} Partie~1 Compr\'ehension -- le 30 novembre
minuit. Partie~2 Construction -- points (a) et (b) -- le 10 d\'ecembre
minuit. Le tout (rapport final complet, i.e. reprenant tout votre travail
(parties 1 et 2 compl\`etes), code source, etc.) -- le 22 d\'ecembre minuit.
 



\subsection{Zone m\'emoire d'un processus de pid $p$}

Les $200$ premiers mots de la tranche m\'emoire pour un un processus de pid
$p$ normal (donc plus grand ou \'egal \`a $1$) sont utilis\'es pour
sauvegarder l'\'etat de l'unit\'e centrale (PSW) lors d'un changement de
contexte. Le \textit{PSW} est le Program Status Word, c'est-\`a-dire la
totalit\'e du contenu des registres CPU~: le pointeur de pile (SP), le
compteur ordinal (PC), les registres g\'en\'eraux (R0, R1, ...), etc. Dans
notre simulateur cette sauvegarde a lieu "automagiquement" lors du
d\'eclenchement d'une interruption~:

\begin{verbatim}
AddrFic  AddrProc   MemProc      Utilisation

- - - - - - - - - - tranche processus p - - - - - - - - -

p*1000        0       1    SP PC MD R0 R1 R2 R3 R4 R5 R6     
p*1000+10    10       1    R7 R8 R9 ....    
p*1000+20    20       1                
   ....      ...      1             ....
p*1000+150   150      1    1  <dsk> <mem> <lgth> ...
   ....      ...      1             ....
p*1000+200   200      1    <debut zone disponible>
   ....      ....     1                ...
p*1000+990   990      1 <start pile (croit vers debut)>

- - - - - - - - - - - - - - - - - - - - - - - - - - - - -
\end{verbatim}

\`A l'adresse $150$ on stocke des renseignements d\'edi\'es \`a l'op\'eration
entr\'ee-sortie courante, lors de son initiation (explication plus bas dans la
section Disque et DMA).

\subsection{Zone m\'emoire du noyau}

Le noyau ne sauvegarde pas son PSW, et utilise le d\'ebut de sa tranche
m\'emoire autrement. Le tout premier entier contient le pid du processus qui
\'etait en train de s'ex\'ecuter juste avant, et ensuite on loge le "vecteur
d'interruptions". 


(donc quatre entiers repr\'esentants les points d'entr\'ee
dans le noyau -- les lignes (donc valeurs de \texttt{PC}) ou commence le code
de chacune des interruptions, ensuite la table des processus (i.e. leurs
\'etats) et ensuite la table des fichiers (i.e. leur pointeur courant).

\begin{verbatim}
Kernel Memory Section
---------------------
0       <crtPID>     2          31        39      62 0 0 0 0 0 
10      0            0          0         0       0  0 0 0 0 0 
20      <procCnt> pid1State  pid2State  pid3State  ...
...
200     0         fd1Pointer fd2Pointer fd3Pointer ...
\end{verbatim}
